\documentclass{amsart}
\input{decls}
\title{Functions}
\author{Frank Tsai}
\date{\today}
%\thanks{}
\begin{document}
\maketitle
\tableofcontents

\section{Countable Sets and Uncountable Sets}
\label{sec:countable-sets-and-uncountable-sets}

\begin{thm}\label{thm:function-space-uncountable}
  $\dN^{\dN}$ is uncountable.
\end{thm}
\begin{proof}
  Suppose that $\dN^{\dN}$ is countable, i.e., $\dN \iso \dN^{\dN}$.
  One possible interpretations of this hypothesis is that every function $f : \dN \to \dN$ can be given a unique natural-number code.
  That is, there are functions
  \begin{align}
    \mathsf{decode} &: \dN \to \dN^{\dN}\\
    \mathsf{encode} &: \dN^{\dN} \to \dN
  \end{align}
  that are mutual inverses.
  Consider the function
  \begin{align}
    k &: \dN \to \dN\\
    k &: n \mapsto \mathsf{decode}(n)(n) + 1
  \end{align}
  Given a code $n$, the function $k$ decodes $n$, yielding a function $\dN \to \dN$, then evaluates that function at $n$, and finally adds 1 to the result.

  The function $k$ has a unique code given by $\mathsf{encode}(k)$.
  Now, let's evaluate $k$ at its own code:
  \begin{align}
    k(\mathsf{encode}(k)) &= \mathsf{decode}(\mathsf{encode}(k))(\mathsf{encode}(k)) + 1\\
                          &= k(\mathsf{encode}(k)) + 1
  \end{align}
  This is a contradiction.\footnote{This proof is not a proof by contradiction. In fact, it is a constructively valid proof.}
\end{proof}

\cref{thm:function-space-uncountable} tells us that some functions $f : \dN \to \dN$ are uncomputable: there are only countably many programs that one can write, but there are uncountably many endofunctions on $\dN$. Thus, some of those functions do not have a corresponding program that computes it.

\end{document}
