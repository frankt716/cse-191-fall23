\documentclass{amsart}
\input{decls}
\title{Functions and Relations}
\author{Frank Tsai}
\date{\today}
%\thanks{}
\begin{document}
\maketitle
\tableofcontents

\section{Relations}
\label{sec:relations}

\begin{defn}
  An $n$-ary \emph{relation} $R$ on a set $S$ is a subset:
  \[
    R \subseteq S^{n}
  \]
  We write $R(a,\ldots,z)$ whenever $(a,\ldots,z) \in R$.
  
  Binary relations will be the main focus of this class.
  For these relations, it is customary to use infix notations.
  That is, we write $aRb$ instead of $R(a,b)$.
\end{defn}

\begin{eg}
  The less-than-or-equal-to relation $\leq$ on $\dN$ is the subset
  \[
    \{(0,0), (0,1), \ldots, (1,1), (1,2), \ldots\} \subseteq \dN \times \dN
  \]
\end{eg}

\begin{eg}
  The divisibility relation $\mid$ on $\dZ$ is defined by
  \[
    a \mid b \iff \exists c.\,b = ac
  \]
  It is the subset
  \[
    \{(a,b) \in \dZ \times \dZ \mid \exists c.\,b = ac\}
  \]
\end{eg}

\begin{eg}
  The adjacency relation in a simple graph: two vertices $u$ are $v$ are adjacent if they are connected by an edge.
  It is the subset
  \[
    \{(u,v) \in V \times V \mid (u, v) \in E \vee (v, u) \in E\}
  \]
\end{eg}

\begin{defn}[Reflexivity]
  A binary relation $R$ on a set $S$ is \emph{reflexive} if for all elements $a$ of $S$, $aRa$.
  \[
    \forall a.\,aRa
  \]
\end{defn}

\begin{defn}[Symmetry]
  A binary relation $R$ on a set $S$ is \emph{symmetric} if for any two elements $a,b$ of $S$, if $aRb$ then $bRa$.
  \[
    \forall a.\,\forall b.\,aRb \imp bRa
  \]
\end{defn}

\section{Countable Sets and Uncountable Sets}
\label{sec:countable-sets-and-uncountable-sets}

\begin{thm}\label{thm:function-space-uncountable}
  $\dN^{\dN}$ is uncountable.
\end{thm}
\begin{proof}
  Suppose that $\dN^{\dN}$ is countable, i.e., $\dN \iso \dN^{\dN}$.
  A possible interpretation of this hypothesis is that every function $f : \dN \to \dN$ can be given a unique natural-number code.
  That is, there are functions
  \begin{align}
    \mathsf{decode} &: \dN \to \dN^{\dN}\\
    \mathsf{encode} &: \dN^{\dN} \to \dN
  \end{align}
  that are mutual inverses.
  Consider the function
  \begin{align}
    k &: \dN \to \dN\\
    k &: n \mapsto \mathsf{decode}(n)(n) + 1
  \end{align}
  Given a code $n$, the function $k$ decodes $n$, yielding a function $\dN \to \dN$, then evaluates that function at $n$, and finally adds 1 to the result.

  The function $k$ has a unique code given by $\mathsf{encode}(k)$.
  Now, let's evaluate $k$ at its own code:
  \begin{align}
    k(\mathsf{encode}(k)) &= \mathsf{decode}(\mathsf{encode}(k))(\mathsf{encode}(k)) + 1\\
                          &= k(\mathsf{encode}(k)) + 1
  \end{align}
  This is a contradiction.
\end{proof}

\cref{thm:function-space-uncountable} tells us that some functions $f : \dN \to \dN$ are uncomputable: there are only countably many programs that one can write, but there are uncountably many endofunctions on $\dN$. Thus, some of those functions do not have a corresponding program that computes it.

\end{document}
