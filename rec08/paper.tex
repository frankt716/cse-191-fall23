\documentclass{amsart}
\input{decls}
\title{Functions and Relations}
\author{Frank Tsai}
\date{\today}
%\thanks{}
\begin{document}
\maketitle
\tableofcontents

\section{Relations}
\label{sec:relations}

\begin{defn}
  An $n$-ary \emph{relation} $R$ on a set $S$ can be encoded as a subset:
  \[
    R \subseteq S^{n}
  \]
  We write $R(a,\ldots,z)$ whenever $(a,\ldots,z) \in R$.
  Binary relations will be the main focus of this class.
  For these relations, it is customary to use infix notations.
  That is, we write $aRb$ instead of $R(a,b)$.
\end{defn}

\begin{eg}
  The substring relation $\sqsubseteq$ on $\{a,b\}^{*}$ is the subset
  \[
    \{(\varepsilon,\varepsilon), (\varepsilon, a), \ldots, (a, a), (a, ab), (a, ba), \ldots\}
  \]
\end{eg}

\begin{eg}
  The divisibility relation $\mid$ on $\dZ$ is defined by
  \[
    a \mid b \iff \exists c.\,b = ac
  \]
  It is the subset
  \[
    \{(a,b) \in \dZ \times \dZ \mid \exists c.\,b = ac\}
  \]
\end{eg}

\begin{eg}
  The adjacency relation on a simple graph: two vertices $u$ are $v$ are adjacent if they are connected by an edge.
  It is the subset
  \[
    \{(u,v) \in V \times V \mid (u, v) \in E \vee (v, u) \in E\}
  \]
\end{eg}

\begin{defn}[Reflexivity]
  A binary relation $R$ on a set $S$ is \emph{reflexive} if every element of $S$ is related to itself by $R$.
  \[
    \forall a.\,aRa
  \]
\end{defn}

\begin{eg}
  The divisibility relation on $\dZ$ is reflexive because every integer divides into itself once.
\end{eg}

\begin{defn}[Symmetry]
  A binary relation $R$ on a set $S$ is \emph{symmetric} if whenever $a$ is related to $b$ by $R$, then $b$ is also related to $a$ by $R$.
  \[
    \forall a.\forall b.\,(aRb \imp bRa)
  \]
\end{defn}

\begin{eg}
  The adjacency relation on a simple graph is symmetric.
  If a vertex $u$ is adjacent to another vertex $v$, then $v$ is also adjacent to $u$.
\end{eg}

\begin{defn}[Transitivity]
  A binary relation $R$ on a set $S$ is \emph{transitive} if for any three elements $a,b,c$ of $S$, if $aRb$ and $bRc$ then $aRc$.
  \[
    \forall a.\forall b.\forall c.\,(aRb \wedge bRc \imp aRc)
  \]
\end{defn}

\begin{eg}
  The substring relation on $\{a,b\}^{*}$ is transitive.
  In fact, it is also reflexive, but it is not symmetric.
\end{eg}

\begin{defn}[Equivalence Relation]
  A binary relation $R$ on a set $S$ is an \emph{equivalence relation} if it is
  \begin{enumerate}
  \item reflexive,
  \item symmetric, and
  \item transitive.
  \end{enumerate}
\end{defn}

\begin{prop}\label{prop:cng-mod-equiv}
  The congruence-modulo-2 relation on $\dZ$ is defined by
  \[
    a \cng b \mod{2} \iff 2 \mid (a - b)
  \]
  It is an equivalence relation.
\end{prop}
\begin{proof}
  (Reflexivity). Let $a$ be any integer.
  We need to prove that $a \cng a \mod{2}$.
  By definition, this is equivalent to proving $2 \mid (a - a)$, or equivalently, $2 \mid 0$.
  By definition again, this is equivalent to $\exists c.\,0 = 2c$.
  Setting $c := 0$ yields $0 = 2 \cdot 0 = 0$ as desired.

  (Symmetry). Let $a, b$ be any integers.
  Assume that $a \cng b \mod{2}$.
  By definition, this hypothesis asserts that there's an integer $c$ so that $a - b = 2c$.
  We need to prove $\exists k.\,b - a = 2k$.
  Setting $k := -c$ yields $b - a = -(a - b) = -2c = 2(-c)$ as desired.

  (Transitivity). Let $a, b, c$ be any integers.
  Assume that $a \cng b \mod{2}$ and that $b \cng c \mod{2}$.
  By definition, these two hypotheses assert that there are integers $n, m$ so that $a - b = 2n$ and $b - c = 2m$.
  We need to show that $\exists k.\,a - c = 2k$.
  Setting $k := n + m$ yields $2(n + m) = 2n + 2m = (a - b) + (b - c) = a - b + b - c = a - c$ as desired.
\end{proof}

\begin{defn}[Antisymmetry]
  A binary relation $R$ on a set $S$ is \emph{antisymmetric} if for any two elements $a, b$ of $S$, if $aRb$ and $bRa$ then $a = b$.
\end{defn}

\begin{eg}
  The subset relation $\subseteq$ on $\cP(S)$ is antisymmetric.
  Recall that two sets $A$ and $B$ are equal precisely when $A \subseteq B$ and $B \subseteq A$.
\end{eg}

\begin{rmk}
  Antisymmetry does \textbf{not} imply \textbf{a}symmetry.
  For example, the indiscrete relation $I$ on the singleton set $\{a\}$, defined as
  \[
    I = \{(a,a)\}
  \]
  is both antisymmetric and symmetric.
\end{rmk}

\begin{defn}[Preorder]
  A binary relation is a \emph{preorder} if it is
  \begin{enumerate}
  \item reflexive, and
  \item transitive.
  \end{enumerate}
\end{defn}

\begin{defn}[Partial Order]
  A \emph{partial order} is a preorder that additionally satisfies antisymmetry.
\end{defn}

\begin{prop}\label{prop:div-n-partial}
  The divisibility relation on $\dN$ is a partial order.
\end{prop}
\begin{proof}
  (Reflexivity): Exercise.

  (Transitivity): Exercise.
  Hint: See \cref{prop:cng-mod-equiv}.

  (Antisymmetry): Let $a, b$ be natural numbers so that $a \mid b$ and $b \mid a$.
  These hypotheses assert that there are natural numbers $n, m$ so that $b = an$ and that $a = bm$.
  Thus, $b = (bm)n$.
  If $b = 0$, then since $a = bm = 0m = 0$, $a = b$ as desired.
  However, if $b \ne 0$, then $mn = 1$.
  Since $n, m$ are natural numbers, $n = m = 1$.
  Thus, $a = b$ as desired.
\end{proof}

\begin{rmk}
  \cref{prop:div-n-partial} does not hold if we replace $\dN$ with $\dZ$ because $2 \mid -2$ and $-2 \mid 2$, but $2 \ne -2$.
  Although the divisibility relation on $\dZ$ is not a partial order, it is a preorder.
\end{rmk}

\section{Functions}
\label{sec:functions}

\section{Countable Sets and Uncountable Sets}
\label{sec:countable-sets-and-uncountable-sets}

\begin{thm}\label{thm:function-space-uncountable}
  $\dN^{\dN}$ is uncountable.
\end{thm}
\begin{proof}
  Suppose that $\dN^{\dN}$ is countable, i.e., $\dN \iso \dN^{\dN}$.
  A possible interpretation of this hypothesis is that every function $f : \dN \to \dN$ can be given a unique natural-number code.
  That is, there are functions
  \begin{align}
    \mathsf{decode} &: \dN \to \dN^{\dN}\\
    \mathsf{encode} &: \dN^{\dN} \to \dN
  \end{align}
  that are mutual inverses.
  Consider the function
  \begin{align}
    k &: \dN \to \dN\\
    k &: n \mapsto \mathsf{decode}(n)(n) + 1
  \end{align}
  Given a code $n$, the function $k$ decodes $n$, yielding a function $\dN \to \dN$, then evaluates that function at $n$, and finally adds 1 to the result.

  The function $k$ has a unique code given by $\mathsf{encode}(k)$.
  Now, let's evaluate $k$ at its own code:
  \begin{align}
    k(\mathsf{encode}(k)) &= \mathsf{decode}(\mathsf{encode}(k))(\mathsf{encode}(k)) + 1\\
                          &= k(\mathsf{encode}(k)) + 1
  \end{align}
  This is a contradiction.
\end{proof}

\cref{thm:function-space-uncountable} tells us that some functions $f : \dN \to \dN$ are uncomputable: there are only countably many programs that one can write, but there are uncountably many endofunctions on $\dN$. Thus, some of those functions do not have a corresponding program that computes it.

\end{document}
