\documentclass{amsart}
\usepackage{agda}
\usepackage{fancyvrb}
\input{decls}
\title{Induction}
\author{Frank Tsai}
\date{\today}
% \thanks{}

\DefineVerbatimEnvironment
{code}{Verbatim}
{}

\begin{document}
\maketitle
\tableofcontents

\section{Induction}
\label{sec:induction}

Roughly speaking, natural numbers $\dN$ are the ``minimal'' data type equipped with a successor function $s : \dN \to \dN$.
We can define $\dN$ inductively:
\begin{enumerate}
\item $0$ is a natural number;
\item if $n$ is a natural number, then $s(n)$ is also a natural number.
\end{enumerate}

Consider the following thought experiment:
If I know $P(0)$ and $P(0) \imp P(1)$, then I can conclude $P(1)$.
\begin{mathpar}
  \inferrule
  { P(0) \\\\ P(0) \imp P(1) }
  { P(1) }
\end{mathpar}
If I additionally know $P(1) \imp P(2)$, then I can also conclude $P(2)$.
\begin{mathpar}
  \inferrule
  { P(0) \\\\ P(0) \imp P(1) \\\\ P(1) \imp P(2) }
  { P(2) }
\end{mathpar}
Now suppose that I know $P(0)$ and $P(0) \imp P(1), P(1) \imp P(2), \ldots$.
In other words, I have the following hypotheses:
\begin{mathpar}
  P(0) \and \forall k.\,(P(k) \imp P(s(k)))
\end{mathpar}
Now, if I want to prove $\forall n.\,P(n)$, I need to introduce a variable $n$ and prove $P(n)$.
Since $n$ is a natural number, it has to be either $0$ or $s(m)$ for some natural number $m$.
If $n$ is $0$, then proving $P(0)$ is easy because it follows directly from the first hypothesis.
If $n$ is $s(m)$, then the second hypothesis says it suffices to prove $P(m)$.
To prove this, we can recursively apply the argument just described, i.e., do a case analysis on $m$.
Each time, the input to $P$ become smaller.

You may have noticed that this is just a recursive function with a base case.
In fact, we can implement this in a programming language called Agda.



\end{document}
