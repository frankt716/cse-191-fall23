\documentclass{amsart}
\input{decls}
\title{Proofs}
\author{Frank Tsai}
\date{\today}
%\thanks{}
\begin{document}
\maketitle
\tableofcontents

\section{Proofs}
\label{sec:proofs}

Logical symbols are explained by (1) how to prove it (introduction) and (2) how to use it (elimination).
We use the capital Greek letter $\Gamma$ to denote a list of hypotheses.
A hypothesis is just a formula.
The order and the number of occurrence of a given formula does not matter.
For example,
\begin{mathpar}
  \varphi, \psi, \chi \and \psi, \varphi, \chi \and \varphi, \varphi, \psi, \chi
\end{mathpar}
are considered to be the same set of hypotheses.

We write
\[
  \Gamma \vdash \varphi
\]
to mean ``the hypotheses in $\Gamma$ \emph{entail} $\varphi$''.

\subsection{Rules}
\label{sec:rules}

The simplest rule is the \emph{identity rule}.
It says that we can conclude $\varphi$ if it is already part of the set of hypotheses.

\begin{mathpar}
  \inferrule[Id]
  { }
  { \Gamma,\varphi \vdash \varphi }
\end{mathpar}

We are going to allow ourselves to do the following rewrites:
\begin{mathpar}
  \neg \varphi \equiv \varphi \imp \bot\and \neg (\neg \varphi) \equiv \varphi
\end{mathpar}

\subsubsection{Top}
\label{sec:top}

We can always derive $\top$, but because we can't use $\top$ in any meaningful way, it has no elimination rules.
\begin{mathpar}
  \inferrule[$\top$-Intro]
  { }
  { \Gamma \vdash \top }
\end{mathpar}

\subsubsection{Conjunction}
\label{sec:conjunction}

Conjunctions behave like pairs.
To construct a pair, we need two elements (the first element and the second element).
It has two elimination rules, one extracts the first element and the other extracts the second element.
\begin{mathpar}
  \inferrule[$\wedge$-Intro]
  { \Gamma \vdash \varphi \\ \Gamma \vdash \psi }
  { \Gamma \vdash \varphi \wedge \psi }\and

  \inferrule[$\wedge$-Elim-L]
  { \Gamma \vdash \varphi \wedge \psi }
  { \Gamma \vdash \varphi }\and

  \inferrule[$\wedge$-Elim-R]
  { \Gamma \vdash \varphi \wedge \psi }
  { \Gamma \vdash \psi }
\end{mathpar}

\paragraph{Proof Script}
\begin{enumerate}
\item To prove $\varphi \wedge \psi$: we need to prove both $\varphi$ and $\psi$.
\item To use $\varphi \wedge \psi$: since $\varphi \wedge \psi$, we may assume both $\varphi$ and $\psi$.
\end{enumerate}

\subsubsection{Implication}
\label{sec:implication}

Implications behave like functions.
If I can write a program $\psi$ with input $\varphi$, then I can abstract the input to get a function.
The elimination rule is also known as \emph{modus ponens}.
It is function application.
\begin{mathpar}
  \inferrule[$\imp$-Intro]
  { \Gamma, \varphi \vdash \psi }
  { \Gamma \vdash \psi \imp \psi }\and

  \inferrule[$\imp$-Elim]
  { \Gamma \vdash \varphi \imp \psi \\ \Gamma \vdash \varphi }
  { \Gamma \vdash \psi }
\end{mathpar}

\paragraph{Proof Script}
\begin{enumerate}
\item To prove $\varphi \imp \psi$: suppose that $\varphi$ ... therefore $\psi$.
\item To use $\varphi \imp \psi$: since $\varphi \imp \psi$, to prove $\psi$ it suffices to prove $\varphi$.
\end{enumerate}

\subsubsection{Universal Quantification}
\label{sec:universal-quantification}

\begin{mathpar}
  \inferrule[$\forall$-Intro]
  { \Gamma \vdash \varphi }
  { \Gamma \vdash \forall x.\varphi }(x \notin \mathrm{FV}(\Gamma))\and

  \inferrule[$\forall$-Elim]
  { \Gamma \vdash \forall x.\varphi }
  { \Gamma \vdash \varphi[t/x] }
\end{mathpar}
Note that the side condition is crucial.
Say, one of the hypotheses is ``$x$ is even'' from which we can prove $x$ is divisible by $2$.
It's incorrect to infer that for all $x$, $x$ is divisible by $2$ because the hypotheses constrain $x$ to even numbers.

\paragraph{Proof Script}
\begin{enumerate}
\item To prove $\forall x. \varphi$: let $x$ be given, ... therefore $\varphi$.
\item To use $\forall x. \varphi$: since $\forall x. \varphi$, $\varphi[t/x]$.
\end{enumerate}

\subsubsection{Bottom}
\label{sec:bottom}

Bottom does not have an introduction rule but it has a powerful elimination rule.
The elimination rule is also known as \emph{ex falso quodlibet}.
\begin{mathpar}
  \inferrule[$\bot$-Elim]
  { \Gamma \vdash \bot }
  { \Gamma \vdash \varphi }
\end{mathpar}

\subsubsection{Disjunction}
\label{sec:disjunction}

Disjunctions behave like tagged unions.
Its elimination rule is the \texttt{if ... then ... else ...} construct in programming languages.
\begin{mathpar}
  \inferrule[$\vee$-Intro-L]
  { \Gamma \vdash \varphi }
  { \Gamma \vdash \varphi \vee \psi }\and

  \inferrule[$\vee$-Intro-R]
  { \Gamma \vdash \psi }
  { \Gamma \vdash \varphi \vee \psi }\and

  \inferrule[$\vee$-Elim]
  { \Gamma \vdash \varphi \vee \psi \\ \Gamma,\varphi \vdash \chi \\ \Gamma,\psi \vdash \chi }
  { \Gamma \vdash \chi }
\end{mathpar}

\paragraph{Proof Script}
\begin{enumerate}
\item To prove $\varphi \vee \psi$: prove either $\varphi$ or $\psi$.
\item To use $\varphi \vee \psi$: case analysis.
\end{enumerate}

\subsubsection{Existential Quantification}
\label{sec:existential-quantification}

\begin{mathpar}
  \inferrule[$\exists$-Intro]
  { \Gamma \vdash \varphi[t/x] }
  { \Gamma \vdash \exists x.\varphi }\and

  \inferrule[$\exists$-Elim]
  { \Gamma \vdash \exists x.\varphi }
  { \Gamma \vdash \varphi[y/x] }(y \notin \mathrm{FV}\Gamma)
\end{mathpar}
The side condition in the elimination rule is crucial because $\exists x.\varphi$ contains a \emph{witness} that we know nothing about so the set of hypotheses $\Gamma$ cannot assume anything about it.

\paragraph{Proof Script}
\begin{enumerate}
\item To prove $\exists x.\varphi$: find a witness $t$ for which $\varphi[t/x]$ holds.
\item To use $\exists x.\varphi$: we get a hypothetical witness $y$ that we don't know anything about and $\varphi[y/x]$ holds.
\end{enumerate}

\section{Examples}
\label{sec:examples}

\begin{eg}\label{eg:contraposition-1}
  \[
    (\varphi \imp \psi) \imp (\neg \psi \imp \neg \varphi)
  \]
\end{eg}
\begin{proof}[Proof 1]
  Suppose $\varphi \imp \psi$.
  We need to prove $\neg \psi \imp \neg \varphi$.
  Suppose $\neg \psi$.
  We need to prove $\neg \varphi$.
  Suppose $\varphi$.
  We need to prove $\bot$.
  Since we know $\neg \psi$, it suffices to prove $\psi$.
  Since we know $\varphi \imp \psi$, it suffices to prove $\varphi$, which is already a hypothesis.
\end{proof}

\begin{proof}[Proof 2]
  \begin{mathpar}
    \inferrule*[Left=$\imp$-Intro]
    { \inferrule*[Left=$\imp$-Intro]
      { \inferrule*[Left=$\imp$-Intro]
        { \inferrule*[Left=$\imp$-Elim]
          { \inferrule*[Left=Id]
            {  }
            { \varphi \imp \psi, \neg \psi, \varphi \vdash \neg \psi }\\
            \Delta
          }
          { \varphi \imp \psi, \neg \psi, \varphi \vdash \bot }
        }
        { \varphi \imp \psi, \neg \psi \vdash \neg \varphi }
      }
      { \varphi \imp \psi \vdash \neg \psi \imp \neg \varphi }
    }
    { \vdash (\varphi \imp \psi) \imp (\neg \psi \imp \neg \varphi) }
  \end{mathpar}
  where $\Delta =$
  \begin{mathpar}
    \inferrule*[Left=$\imp$-Elim]
    { \inferrule*[left=Id]
      {  }
      { \varphi \imp \psi, \neg \psi, \varphi \vdash \varphi \imp \psi }\\
      \inferrule*[Right=Id]
      {  }
      { \varphi \imp \psi, \neg \psi, \varphi \vdash \varphi }
    }
    { \varphi \imp \psi, \neg \psi, \varphi \vdash \psi }
  \end{mathpar}
\end{proof}

\begin{eg}\label{eg:contraposition-2}
  \[
    (\neg \psi \imp \neg \varphi) \imp (\varphi \imp \psi)
  \]
\end{eg}
\begin{proof}[Proof 1]
  Exercise.
\end{proof}
\begin{proof}[Proof 2]
  \begin{mathpar}
    \inferrule*[Left=$\imp$-Intro]
    { \inferrule*[Left=$\imp$-Intro]
      { \inferrule*[Left=$\imp$-Intro]
        { \inferrule*[Left=$\imp$-Elim]
          { \Delta\\
            \inferrule*[Right=$\imp$-Elim]
            { \inferrule*[left=Id]
              {  }
              { \neg \psi \imp \neg \varphi, \neg (\neg \varphi), \neg \psi \vdash \neg \psi \imp \neg \varphi }\\
              \Xi
            }
            { \neg \psi \imp \neg \varphi, \neg (\neg \varphi), \neg \psi \vdash \neg \varphi }
          }
          { \neg \psi \imp \neg \varphi, \neg (\neg \varphi), \neg \psi \vdash \bot }
        }
        { \neg \psi \imp \neg \varphi, \varphi \vdash \neg (\neg \psi) }
      }
      { \neg \psi \imp \neg \varphi \vdash \varphi \imp \psi }
    }
    { \vdash (\neg \psi \imp \neg \varphi) \imp (\varphi \imp \psi) }
  \end{mathpar}
  where $\Delta =$
  \begin{mathpar}
    \inferrule*[Left=Id]
    {  }
    { \neg \psi \imp \neg \varphi, \neg (\neg \varphi), \neg \psi \vdash \neg (\neg \varphi) }
  \end{mathpar}
  and $\Xi =$
  \begin{mathpar}
    \inferrule*[Left=Id]
    {  }
    { \neg \psi \imp \neg \varphi, \neg (\neg \varphi), \neg \psi \vdash \neg \psi }
  \end{mathpar}
\end{proof}

\begin{eg}
  \[
    \neg (\varphi \vee \psi) \imp \neg \varphi \wedge \neg \psi
  \]
\end{eg}
\begin{proof}[Proof 1]
  Suppose that $\neg (\varphi \vee \psi)$.
  We need to prove $\neg \varphi \wedge \neg \psi$.
  It suffices to prove $\neg \varphi$ and $\neg \psi$ separately.

  \begin{enumerate}
  \item Proof of $\neg \varphi$: Assume $\varphi$, we need to prove $\bot$.
    Since we know that $\neg (\varphi \vee \psi)$, it suffices to prove $\varphi \vee \psi$, which follows immediately from $\varphi$.
  \item Proof of $\neg \psi$: Assume $\psi$, we need to prove $\bot$.
    Since we know that $\neg (\varphi \vee \psi)$, it suffices to prove $\varphi \vee \psi$, which follows immediately from $\psi$.
  \end{enumerate}
\end{proof}

\begin{proof}[Proof 2]
  \begin{mathpar}
    \inferrule*[left=$\imp$-Intro]
    { \inferrule*[Left=$\wedge$-Intro]
      { \inferrule*[Left=$\imp$-Intro]
        { \inferrule*[Left=$\imp$-Elim]
          { \inferrule*[Left=Id]
            {  }
            { \neg (\varphi \vee \psi), \varphi \vdash \neg (\varphi \vee \psi) }\\
            \Delta
          }
          { \neg (\varphi \vee \psi), \varphi \vdash \bot }
        }
        { \neg (\varphi \vee \psi) \vdash \neg \varphi }\\
        \inferrule*[Right=$\imp$-Intro]
        { \Xi }
        { \neg (\varphi \vee \psi) \vdash \neg \psi }
      }
      { \neg (\varphi \vee \psi) \vdash \neg \varphi \wedge \neg \psi }
    }
    { \vdash \neg (\varphi \vee \psi) \imp \neg \varphi \wedge \neg \psi }
  \end{mathpar}
  where $\Delta =$
  \begin{mathpar}
    \inferrule*[Left=$\vee$-Intro-L]
    { \inferrule*[Left=Id]
      {  }
      { \neg (\varphi \vee \psi), \varphi \vdash \varphi }
    }
    { \neg (\varphi \vee \psi), \varphi \vdash \varphi \vee \psi }
  \end{mathpar}
  and $\Xi =$
  \begin{mathpar}
    \inferrule*[Left=$\imp$-Elim]
    { \inferrule*[Left=Id]
      {  }
      { \neg (\varphi \vee \psi), \psi \vdash \neg (\varphi \vee \psi) }\\
      \inferrule*[Right=$\vee$-Intro-R]
      { \inferrule*[Right=Id]
        {  }
        { \neg (\varphi \vee \psi), \psi \vdash \psi }
      }
      { \neg (\varphi \vee \psi), \psi \vdash \varphi \vee \psi }
    }
    { \neg (\varphi \vee \psi), \psi \vdash \bot }
  \end{mathpar}
\end{proof}

\begin{eg}
  \[
    \forall x.\varphi \imp \exists x.\varphi
  \]
\end{eg}
\begin{proof}[Proof 1]
  Suppose that $\forall x.\varphi$.
  We need to prove $\exists x.\varphi$.
  It suffices to find a witness $t$ so that $\varphi[t/x]$.
  Since we know $\forall x.\varphi$, any witness works.
\end{proof}

\begin{proof}[Proof 2]
  \begin{mathpar}
    \inferrule*[Left=$\imp$-Intro]
    { \inferrule*[Left=$\exists$-Intro]
      { \inferrule*[Left=$\forall$-Elim]
        { \inferrule*[Left=Id]
          {  }
          { \forall x.\varphi \vdash \forall x.\varphi }
        }
        { \forall x.\varphi \vdash \varphi[t/x] }
      }
      { \forall x.\varphi \vdash \exists x.\varphi }
    }
    { \vdash \forall x.\varphi \imp \exists x.\varphi }
  \end{mathpar}
\end{proof}

\begin{eg}
  \[
    \neg (\exists x.\varphi) \imp \forall x.\neg \varphi
  \]
\end{eg}
\begin{proof}[Proof 1]
  Suppose $\neg \exists x.\varphi$.
  We need to prove $\forall x.\neg \varphi$.
  Let any $x$ be given.
  We need to prove $\neg \varphi$.
  Suppose $\varphi$.
  We need to prove $\bot$.
  Since we know $\neg \exists x.\varphi$, it suffices to prove $\exists x.\varphi$.
  Since we already know $\varphi$, we can choose $x$ as the witness.
\end{proof}
\begin{proof}[Proof 2]
  \begin{mathpar}
    \inferrule*[Left=$\imp$-Intro]
    { \inferrule*[Left=$\imp$-Intro]
      { \inferrule*[Left=$\imp$-Intro]
        { \inferrule*[Left=$\imp$-Elim]
          { \inferrule*[Left=Id]
            {  }
            { \neg (\exists x.\varphi), \varphi \vdash \neg (\exists x.\varphi) }\\
            \inferrule*[Right=$\exists$-Intro]
            { \inferrule*[Right=Id]
              {  }
              { \neg (\exists x.\varphi), \varphi \vdash \varphi[x/x] }
            }
            { \neg (\exists x.\varphi), \varphi \vdash \exists x.\varphi }
          }
          { \neg (\exists x.\varphi), \varphi \vdash \bot }
        }
        { \neg (\exists x.\varphi) \vdash \neg \varphi }
      }
      { \neg (\exists x.\varphi) \vdash \forall x.\neg \varphi }
    }
    { \vdash \neg (\exists x.\varphi) \imp \forall x.\neg \varphi }
  \end{mathpar}
\end{proof}

\begin{eg}
  \[
    \neg (\forall x. \varphi) \imp \exists x. \neg \varphi
  \]
\end{eg}
\begin{proof}[Proof 1]
  Exercise.
\end{proof}
\begin{proof}[Proof 2]
  \begin{mathpar}
    \inferrule*[Left=$\imp$-Elim]
    { \Delta\\
      \inferrule*[Right=$\imp$-Intro]
      { \inferrule*[Right=$\imp$-Intro]
        { \inferrule*[Right=$\imp$-Intro]
          { \inferrule*[Right=$\imp$-Elim]
            { \inferrule*[Left=Id]
              {  }
              { \neg \exists x.\neg \varphi, \neg \varphi \vdash \neg \exists x.\neg \varphi }\\
              \inferrule*[Right=$\exists$-Intro]
              { \inferrule*[Right=Id]
                {  }
                { \neg \exists x.\neg \varphi, \neg \varphi \vdash \neg \varphi[x/x] }
              }
              { \neg \exists x.\neg \varphi, \neg \varphi \vdash \exists x. \neg \varphi }
            }
            { \neg \exists x.\neg \varphi, \neg \varphi \vdash \bot }
          }
          { \neg \exists x.\neg \varphi \vdash \neg (\neg \varphi) }
        }
        { \neg \exists x.\neg \varphi \vdash \forall x.\varphi }
      }
      { \vdash \neg \exists x.\neg \varphi \imp \forall x.\varphi }
    }
    { \vdash \neg (\forall x. \varphi) \imp \exists x. \neg \varphi }
  \end{mathpar}
  where $\Delta =$
  \begin{mathpar}
    \inferrule*[]
    { \vdots }
    { \vdash (\neg (\exists x.\neg \varphi) \imp \forall x.\varphi) \imp \neg (\forall x. \varphi) \imp \exists x. \neg \varphi }
  \end{mathpar}
  follows from \cref{eg:contraposition-2}.
\end{proof}

\end{document}
