\documentclass{amsart}
\input{decls}
\title{Natural Deduction}
\author{Frank Tsai}
\date{\today}
%\thanks{}
\begin{document}
\maketitle
\tableofcontents

\section{Natural Deduction}
\label{sec:natural-deduction}

Logical symbols are explained by (1) how to prove it (introduction) and (2) how to use it (elimination).
We use the capital Greek letter $\Gamma$ to denote a list of hypotheses.
A hypothesis is just a formula.
The order and the number of occurrence of a given formula does not matter.
For example,
\begin{mathpar}
  \varphi, \psi, \chi \and \psi, \varphi, \chi \and \varphi, \varphi, \psi, \chi
\end{mathpar}
are considered to be the set of hypotheses.

We write
\[
  \Gamma \vdash \varphi
\]
to mean ``the hypotheses in $\Gamma$ \emph{entail} $\varphi$''.

\subsection{Rules}
\label{sec:rules}

The simplest rule is the \emph{identity rule}.
It says that we can conclude $\varphi$ if it is already part of the set of hypotheses.

\begin{mathpar}
  \inferrule[Id]
  { }
  { \Gamma,\varphi \vdash \varphi }
\end{mathpar}

\subsubsection{Top}
\label{sec:top}

We can always derive $\top$, but because we can't use $\top$ in any meaningful way, it has no elimination rules.
\begin{mathpar}
  \inferrule[$\top$-Intro]
  { }
  { \Gamma \vdash \top }
\end{mathpar}

\subsubsection{Conjunction}
\label{sec:conjunction}

Conjunctions behave like pairs.
To construct a pair, we need two elements (the first element and the second element).
It has two elimination rules, one extracts the first element and the other extracts the second element.
\begin{mathpar}
  \inferrule[$\wedge$-Intro]
  { \Gamma \vdash \varphi \\ \Gamma \vdash \psi }
  { \Gamma \vdash \varphi \wedge \psi }\and

  \inferrule[$\wedge$-Elim-L]
  { \Gamma \vdash \varphi \wedge \psi }
  { \Gamma \vdash \varphi }\and

  \inferrule[$\wedge$-Elim-R]
  { \Gamma \vdash \varphi \wedge \psi }
  { \Gamma \vdash \psi }
\end{mathpar}

\paragraph{Proof Script}
\begin{enumerate}
\item To prove $\varphi \wedge \psi$: we need to prove both $\varphi$ and $\psi$.
\item To use $\varphi \wedge \psi$: since $\varphi \wedge \psi$, we may assume both $\varphi$ and $\psi$.
\end{enumerate}

\subsubsection{Implication}
\label{sec:implication}

Implications behave like functions.
If I can write a program $\psi$ with input $\varphi$, then I can abstract the input to get a function.
The elimination rule is also known as \emph{modus ponens}.
It is function application.
\begin{mathpar}
  \inferrule[$\imp$-Intro]
  { \Gamma, \varphi \vdash \psi }
  { \Gamma \vdash \psi \imp \psi }\and

  \inferrule[$\imp$-Elim]
  { \Gamma \vdash \varphi \imp \psi \\ \Gamma \vdash \varphi }
  { \Gamma \vdash \psi }
\end{mathpar}

\paragraph{Proof Script}
\begin{enumerate}
\item To prove $\varphi \imp \psi$: suppose that $\varphi$ ... therefore $\psi$.
\item To use $\varphi \imp \psi$: since $\varphi \imp \psi$, to prove $\psi$ it suffices to prove $\varphi$.
\end{enumerate}

\subsubsection{Universal Quantification}
\label{sec:universal-quantification}

\begin{mathpar}
  \inferrule[$\forall$-Intro]
  { \Gamma \vdash \varphi }
  { \Gamma \vdash \forall x.\varphi }(x \notin \Gamma)\and

  \inferrule[$\forall$-Elim]
  { \Gamma \vdash \forall x.\varphi }
  { \Gamma \vdash \varphi[t/x] }
\end{mathpar}
Note that the side condition is crucial.
Say, one of the hypotheses is ``$x$ is even'' from which we can prove $x$ is divisible by $2$.
It's incorrect to infer that for all $x$, $x$ is divisible by $2$ because the hypotheses constrain $x$ to even numbers.

\paragraph{Proof Script}
\begin{enumerate}
\item To prove $\forall x. \varphi$: let $x$ be given, ... therefore $\varphi$.
\item To use $\forall x. \varphi$: since $\forall x. \varphi$, $\varphi[t/x]$.
\end{enumerate}

\subsubsection{Bottom}
\label{sec:bottom}

Bottom does not have an introduction rule but it has a powerful elimination rule.
The elimination rule is also known as \emph{ex falso quodlibet}.
\begin{mathpar}
  \inferrule[$\bot$-Elim]
  { \Gamma \vdash \bot }
  { \Gamma \vdash \varphi }
\end{mathpar}

\subsubsection{Disjunction}
\label{sec:disjunction}

Disjunctions behave like tagged unions.
Its elimination rule is the \texttt{if ... then ... else ...} construct in programming languages.
\begin{mathpar}
  \inferrule[$\vee$-Intro-L]
  { \Gamma \vdash \varphi }
  { \Gamma \vdash \varphi \vee \psi }\and

  \inferrule[$\vee$-Intro-R]
  { \Gamma \vdash \psi }
  { \Gamma \vdash \varphi \vee \psi }\and

  \inferrule[$\vee$-Elim]
  { \Gamma \vdash \varphi \vee \psi \\ \Gamma,\varphi \vdash \chi \\ \Gamma,\psi \vdash \chi }
  { \Gamma \vdash \chi }
\end{mathpar}

\paragraph{Proof Script}
\begin{enumerate}
\item To prove $\varphi \vee \psi$: prove either $\varphi$ or $\psi$.
\item To use $\varphi \vee \psi$: case analysis.
\end{enumerate}

\subsubsection{Existential Quantification}
\label{sec:existential-quantification}

\begin{mathpar}
  \inferrule[$\exists$-Intro]
  { \Gamma \vdash \varphi[t/x] }
  { \Gamma \vdash \exists x.\varphi }\and

  \inferrule[$\exists$-Elim]
  { \Gamma \vdash \exists x.\varphi }
  { \Gamma \vdash \varphi[y/x] }(y \notin \Gamma)
\end{mathpar}
The side condition in the elimination rule is crucial because $\exists x.\varphi$ contains a \emph{witness} that we know nothing about so the set of hypotheses $\Gamma$ cannot assume anything about it.

\paragraph{Proof Script}
\begin{enumerate}
\item To prove $\exists x.\varphi$: find a witness $t$ for which $\varphi[t/x]$ holds.
\item To use $\exists x.\varphi$: we get a hypothetical witness $y$ that we don't know anything about and $\varphi[y/x]$ holds.
\end{enumerate}

\section{Examples}
\label{sec:examples}

\begin{eg}
  \[
    \varphi \wedge \psi \imp \varphi
  \]
\end{eg}
\begin{proof}[Proof 1]
  Suppose that $\varphi \wedge \psi$, then $\varphi$ and $\psi$.
  We need to prove $\varphi$ but it is already a hypothesis.
\end{proof}

\begin{proof}[Proof 2]
  \begin{mathpar}
    \inferrule*[left=$\imp$-Intro]
    { \inferrule*[Left=$\wedge$-Elim-L]
      { \inferrule*[Left=Id]
        {  }
        { \varphi \wedge \psi \vdash \varphi \wedge \psi }
      }
      { \varphi \wedge \psi \vdash \varphi }
    }
    { \vdash \varphi \wedge \psi \imp \varphi }
  \end{mathpar}
\end{proof}

\begin{eg}
  \[
    \neg (\varphi \vee \psi) \imp \neg \varphi \wedge \neg \psi
  \]
\end{eg}
\begin{proof}[Proof 1]
  Suppose that $\neg (\varphi \vee \psi)$.
  We need to prove $\neg \varphi \wedge \neg \psi$.
  It suffices to prove $\neg \varphi$ and $\neg \psi$ separately.

  \begin{enumerate}
  \item Proof of $\neg \varphi$: Assume $\varphi$, we need to prove $\bot$.
    Since we know that $\neg (\varphi \vee \psi)$, it suffices to prove $\varphi \vee \psi$, which follows immediately from $\varphi$.
  \item Proof of $\neg \psi$: Assume $\psi$, we need to prove $\bot$.
    Since we know that $\neg (\varphi \vee \psi)$, it suffices to prove $\varphi \vee \psi$, which follows immediately from $\psi$.
  \end{enumerate}
\end{proof}

\begin{proof}[Proof 2]
  \begin{mathpar}
    \inferrule*[left=$\imp$-Intro]
    { \inferrule*[Left=$\wedge$-Intro]
      { \inferrule*[Left=$\imp$-Intro]
        { \inferrule*[Left=$\imp$-Elim]
          { \inferrule*[Left=Id]
            {  }
            { \neg (\varphi \vee \psi), \varphi \vdash \neg (\varphi \vee \psi) }\\
            \Delta
          }
          { \neg (\varphi \vee \psi), \varphi \vdash \bot }
        }
        { \neg (\varphi \vee \psi) \vdash \neg \varphi }\\
        \inferrule*[Right=$\imp$-Intro]
        { \Xi }
        { \neg (\varphi \vee \psi) \vdash \neg \psi }
      }
      { \neg (\varphi \vee \psi) \vdash \neg \varphi \wedge \neg \psi }
    }
    { \vdash \neg (\varphi \vee \psi) \imp \neg \varphi \wedge \neg \psi }
  \end{mathpar}
  where $\Delta =$
  \begin{mathpar}
    \inferrule*[Left=$\vee$-Intro-L]
    { \inferrule*[Left=Id]
      {  }
      { \neg (\varphi \vee \psi), \varphi \vdash \varphi }
    }
    { \neg (\varphi \vee \psi), \varphi \vdash \varphi \vee \psi }
  \end{mathpar}
  and $\Xi =$
  \begin{mathpar}
    \inferrule*[Left=$\imp$-Elim]
    { \inferrule*[Left=Id]
      {  }
      { \neg (\varphi \vee \psi), \psi \vdash \neg (\varphi \vee \psi) }\\
      \inferrule*[Right=$\vee$-Intro-R]
      { \inferrule*[Right=Id]
        {  }
        { \neg (\varphi \vee \psi), \psi \vdash \psi }
      }
      { \neg (\varphi \vee \psi), \psi \vdash \varphi \vee \psi }
    }
    { \neg (\varphi \vee \psi), \psi \vdash \bot }
  \end{mathpar}
\end{proof}

\begin{eg}
  \[
    \forall x.\varphi \imp \exists x.\varphi
  \]
\end{eg}
\begin{proof}[Proof 1]
  Suppose that $\forall x.\varphi$.
  We need to prove $\exists x.\varphi$.
  It suffices to find a witness $t$ so that $\varphi[t/x]$.
  Since we know $\forall x.\varphi$, any witness works.
\end{proof}

\begin{proof}[Proof 2]
  \begin{mathpar}
    \inferrule*[Left=$\imp$-Intro]
    { \inferrule*[Left=$\exists$-Intro]
      { \inferrule*[Left=$\forall$-Elim]
        { \inferrule*[Left=Id]
          {  }
          { \forall x.\varphi \vdash \forall x.\varphi }
        }
        { \forall x.\varphi \vdash \varphi[t/x] }
      }
      { \forall x.\varphi \vdash \exists x.\varphi }
    }
    { \vdash \forall x.\varphi \imp \exists x.\varphi }
  \end{mathpar}
\end{proof}

\end{document}
