\documentclass{amsart}
\input{decls}
\title{Proof Calculus}
\author{Frank Tsai}
\date{\today}
%\thanks{}
\begin{document}
\maketitle
\tableofcontents

\section{Proof Calculus}
\label{sec:proof-calculus}

Logical symbols are explained by (1) how to prove it (introduction) and (2) how to use it (elimination).
We use the capital Greek letter $\Gamma$ to denote a list of hypotheses.
A hypothesis is just a formula.
The order and the number of occurrence of a given formula does not matter.
For example,
\begin{mathpar}
  \varphi, \psi, \chi \and \psi, \varphi, \chi \and \varphi, \varphi, \psi, \chi
\end{mathpar}
are considered to be the set of hypotheses.

We write
\[
  \Gamma \vdash \varphi
\]
to mean ``the hypotheses in $\Gamma$ \emph{entail} $\varphi$''.

\subsection{Rules}
\label{sec:rules}

The simplest rule is the \emph{identity rule}.
It says that we can conclude $\varphi$ if it is already part of the set of hypotheses.

\begin{mathpar}
  \inferrule[Id]
  { }
  { \Gamma,\varphi \vdash \varphi }
\end{mathpar}

\subsubsection{Top}
\label{sec:top}

\begin{mathpar}
  \inferrule[$\top$-Intro]
  { }
  { \Gamma \vdash \top }
\end{mathpar}

\subsubsection{Conjunction}
\label{sec:conjunction}

\begin{mathpar}
  \inferrule[$\wedge$-Intro]
  { \Gamma \vdash \varphi \\ \Gamma \vdash \psi }
  { \Gamma \vdash \varphi \wedge \psi }\and

  \inferrule[$\wedge$-Elim-L]
  { \Gamma \vdash \varphi \wedge \psi }
  { \Gamma \vdash \varphi }\and

  \inferrule[$\wedge$-Elim-R]
  { \Gamma \vdash \varphi \wedge \psi }
  { \Gamma \vdash \psi }
\end{mathpar}

\subsubsection{Implication}
\label{sec:implication}

\begin{mathpar}
  \inferrule[$\imp$-Intro]
  { \Gamma, \varphi \vdash \psi }
  { \Gamma \vdash \psi \imp \psi }\and

  \inferrule[$\imp$-Elim]
  { \Gamma \vdash \varphi \imp \psi \\ \Gamma \vdash \varphi }
  { \Gamma \vdash \psi }
\end{mathpar}

\subsubsection{Universal Quantification}
\label{sec:universal-quantification}

\begin{mathpar}
  \inferrule[$\forall$-Intro]
  { \Gamma \vdash \varphi }
  { \Gamma \vdash \forall x.\varphi }(x \notin \Gamma)\and

  \inferrule[$\forall$-Elim]
  { \Gamma \vdash \forall x.\varphi }
  { \Gamma \vdash \varphi[t/x] }
\end{mathpar}

\subsubsection{Bottom}
\label{sec:bottom}

\begin{mathpar}
  \inferrule[$\bot$-Elim]
  { \Gamma \vdash \bot }
  { \Gamma \vdash \varphi }
\end{mathpar}

\subsubsection{Disjunction}
\label{sec:disjunction}

\begin{mathpar}
  \inferrule[$\vee$-Intro-L]
  { \Gamma \vdash \varphi }
  { \Gamma \vdash \varphi \vee \psi }\and

  \inferrule[$\vee$-Intro-R]
  { \Gamma \vdash \psi }
  { \Gamma \vdash \varphi \vee \psi }\and

  \inferrule[$\vee$-Elim]
  { \Gamma \vdash \varphi \vee \psi \\ \Gamma,\varphi \vdash \chi \\ \Gamma,\psi \vdash \chi }
  { \Gamma \vdash \chi }
\end{mathpar}

\subsubsection{Existential Quantification}
\label{sec:existential-quantification}

\begin{mathpar}
  \inferrule[$\exists$-Intro]
  { \Gamma \vdash \varphi[t/x] }
  { \Gamma \vdash \exists x.\varphi }\and

  \inferrule[$\exists$-Elim]
  { \Gamma \vdash \exists x.\varphi }
  { \Gamma \vdash \varphi[y/x] }(y \notin \Gamma)
\end{mathpar}

\end{document}
