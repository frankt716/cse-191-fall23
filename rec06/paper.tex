\documentclass{amsart}
\input{decls}
\title{Sets}
\author{Frank Tsai}
\date{\today}
%\thanks{}
\begin{document}
\maketitle
\tableofcontents

\section{Set Theory}
\label{sec:basic-set-theory}

Set theory made its debut in Cantor's 1874 paper: ``On a Property of the Collection of All Real Algebraic Numbers.''
The Cantorian set theory contained several contradictions such as Russell's paradox, but mathematicians did not abandon set theory entirely.
The work of Zermelo, Fraenkel, and Skolem resulted in a well-known axiomatizations of set theory.

In this class, we will not talk about axiomatic set theory.
Instead, we will look at set theory na\"ively.

\subsection{Set}
\label{sec:set}

The atomic objects in set theory are sets.
A set can be thought of as a collection of dots.
These dots have no internal structures, so the only sensible question that one can ask is ``is $x$ a member of $y$?''

The language of set theory contains a binary predicate symbol $\in$.
This is known as the \emph{membership relation}.
$x \in y$ means $x$ is an element of $y$.

\subsection{Subsets}
\label{sec:subsets}

\begin{defn}
  A set $x$ is a subset of $y$, denoted $x \subseteq y$, if every element in $x$ is also in $y$.
  \[
    x \subseteq y \cng \forall z.(z \in x \imp z \in y)
  \]
  The relation $\subseteq$ is called \emph{set inclusion}.
\end{defn}

\subsection{Equality}
\label{sec:equality}

Two sets are equal when they contain the same elements.
We can express this in terms of the set inclusion relation.
\[
  \forall x. \forall y. ((x \subseteq y \wedge y \subseteq x) \imp x = y)
\]
Given two sets $x$ and $y$, to prove that $x = y$, it suffices to prove $x \subseteq y$ and $y \subseteq x$.

\subsection{Comprehension}
\label{sec:comprehension}

Given a set $w$, there is a \emph{subcollection} of $w$ whose elements satisfy certain property $\varphi$.
\[
  \{ x \in w \mid \varphi(x) \}
\]
We declare that such subcollections are sets.
For example, given the set of all natural numbers $\dN$,
\[
  \{ x \in \dN \mid \mathrm{even}(x) \}
\]
is a subcollection of all even natural numbers.

\subsection{Union}
\label{sec:union}

\subsection{Intersection}
\label{sec:intersection}

\subsection{von Neumann Ordinals}
\label{sec:von-neumann-ordinals}

\end{document}
