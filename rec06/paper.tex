\documentclass{amsart}
\input{decls}
\title{Sets}
\author{Frank Tsai}
\date{\today}
%\thanks{}
\begin{document}
\maketitle
\tableofcontents

\section{Set Theory}
\label{sec:basic-set-theory}

Set theory made its debut in Cantor's 1874 paper: ``On a Property of the Collection of All Real Algebraic Numbers.''
Later, mathematicians such as Russell realized that the Cantorian set theory contained several contradictions, but mathematicians did not abandon set theory.
The work of Zermelo, Fraenkel, and Skolem resulted in a well-known axiomatization of set theory.

In this class, we will not talk about axiomatic set theory.
Instead, we will look at set theory na\"ively.

\subsection{Set}
\label{sec:set}

Intuitively, a set is a collection of elements.
These elements have no internal structures, so you can think of a set as a collection of dots, and the only question that one can ask is ``is $x$ a member of $y$?''
In other words, the language of set theory contains a binary predicate symbol $\in$.
This is known as the \emph{membership relation}.
$x \in y$ means $x$ is an element of $y$.

\begin{eg}
  \begin{enumerate}
  \item[]
  \item The empty set: $\varnothing$.
  \item The set containing the empty set: $\{\varnothing\}$.
  \item A set containing 3 elements: $\{ a,b,c \}$.
  \item The set of all natural numbers: $\dN = \{ 0,1,2,\ldots \}$.
  \item The set of all integers: $\dZ = \{ \ldots,-2,-1,0,1,2,\ldots \}$.
  \item The set containing $\dN$ and $\dZ$: $\{ \dN,\dZ \}$.
  \end{enumerate}
\end{eg}

\begin{eg}
  \begin{enumerate}
  \item[]
  \item Nothing is in the empty set: $x \notin \varnothing$.
  \item The set containing the empty set has an element: $\varnothing \in \{\varnothing\}$.
  \item $a \in \{ a, b, c \}$, $b \in \{ a, b, c \}$, $c \in \{ a, b, c \}$.
  \item $\dN \in \{ \dN, \dZ \}$, $\dZ \in \{ \dN, \dZ \}$, but $0 \notin \{ \dN,\dZ \}$.
  \end{enumerate}
\end{eg}

\subsection{Subsets}
\label{sec:subsets}

\begin{defn}
  A set $x$ is a subset of $y$, denoted $x \subseteq y$, if every element in $x$ is also in $y$.
  \[
    x \subseteq y \cng \forall z.(z \in x \imp z \in y)
  \]
  The relation $\subseteq$ is called \emph{set inclusion}.
\end{defn}

\begin{lem}
  The empty set is a subset of any set.
  \[
    \forall y.\,\varnothing \subseteq y
  \]
\end{lem}
\begin{proof}
  Let $y$ be any set.
  By definition, $\varnothing \subseteq y \cng \forall z.(z \in \varnothing \imp z \in y)$.
  Let $z$ be given.
  Assume that $z \in \varnothing$, but this is impossible since $z \notin \varnothing$.\footnote{This is not a proof by contradiction, but let's ignore the details for now.}
\end{proof}

\begin{lem}
  Every set is a subset of itself.
\end{lem}
\begin{proof}
  Exercise.
\end{proof}

\subsection{Equality}
\label{sec:equality}

Two sets are equal when they contain the same elements.
We can express this in terms of the set inclusion relation.
\[
  \forall x. \forall y. ((x \subseteq y \wedge y \subseteq x) \imp x = y)
\]
Given two sets $x$ and $y$, to prove that $x = y$, it suffices to prove $x \subseteq y$ and $y \subseteq x$.

\begin{eg}
  \begin{enumerate}
  \item[]
  \item $\{a,b,c,d,d\} = \{a,b,c,d\}$.
  \item $\{a,b,c\} = \{c,b,a\}$.
  \end{enumerate}
\end{eg}

\subsection{Comprehension}
\label{sec:comprehension}

Given a set $w$, there is a \emph{subcollection} of $w$ whose elements satisfy certain property $\varphi$.
\[
  \{ x \in w \mid \varphi(x) \}
\]
We declare that such subcollections are sets.
For example, given the set of all natural numbers $\dN$,
\[
  \{ x \in \dN \mid \mathrm{even}(x) \}
\]
is the subset of all even natural numbers.

\subsection{Union}
\label{sec:union}

\subsection{Intersection}
\label{sec:intersection}

\subsection{Set Difference}
\label{sec:set-difference}



\subsection{von Neumann Ordinals}
\label{sec:von-neumann-ordinals}

\end{document}
